\thispagestyle{thachthuctoanhocnone}
\pagestyle{thachthuctoanhoc}
\everymath{\color{thachthuctoanhoc}}
\graphicspath{{../thachthuctoanhoc/pic/}}
\begingroup
\AddToShipoutPicture*{\put(0,616){\includegraphics[width=19.3cm]{../thachthuctoanhoc/bannerthachthuc}}}
\centering
\vspace*{4cm}
\endgroup
\vspace*{-8pt}
\begin{tBox}
	\begin{itemize}[leftmargin = 13pt, itemsep = 1.0pt] 
		\item Mỗi bài toán đề xuất (kèm theo lời giải) cần được nêu rõ là bài sáng tác hay bài sưu tầm.
		%		\item Mỗi bài toán đề xuất (kèm theo lời giải) cần được nêu rõ là bài sáng tác hay bài sưu tầm (nếu là bài sưu tầm, cần ghi rõ nguồn).
		\item Bài giải cho mỗi bài toán cần được trình bày trong một file riêng hoặc
		một tờ giấy riêng.
		\item  Người đề xuất bài toán hoặc gửi bài giải cho các bài toán trong mục ``Thách thức kỳ này" cần ghi rõ họ, đệm, tên và nơi làm việc/học tập, số điện thoại liên hệ. Nếu là học sinh (hoặc sinh viên) cần ghi rõ là học sinh lớp mấy (hoặc sinh viên năm thứ mấy).
		\item Các bài toán trong mục Thách thức kỳ này hướng tới các độc giả là học sinh phổ thông; được phân chia thành các mức độ $B$, $A$, và được sắp xếp theo độ khó tăng dần, theo đánh giá chủ quan của Ban biên tập. Các bài toán mức độ $B$ không đòi hỏi các kiến thức vượt quá chương trình môn Toán cấp THCS; các bài toán mức độ $A$ không đòi hỏi các kiến thức vượt quá chương trình môn Toán cấp THPT.
		\item Cách thức gửi bài toán đề xuất hoặc lời giải: gửi file thu được bằng cách scan, ảnh chụp (rõ nét) của bản viết tay, hoặc được soạn thảo bằng các phần mềm Latex, Word tới \url{bbt@pi.edu.vn} hoặc gửi qua đường bưu điện tới Tòa soạn (xem địa chỉ tại bìa $2$).
		\item Hạn gửi lời giải cho các bài toán P$731$--P$740$: trước ngày $15/10/2023$.
	\end{itemize}
\end{tBox}
\begin{center}
	\vspace*{-5pt}
	\textbf{\color{thachthuctoanhoc}\color{thachthuctoanhoc}\color{thachthuctoanhoc}\color{thachthuctoanhoc}THÁCH THỨC KỲ NÀY}
	\vspace*{-5pt}
\end{center}
\begin{multicols}{2}
	\setlength{\abovedisplayskip}{4pt}
	\setlength{\belowdisplayskip}{4pt}
	{\color{thachthuctoanhoc}{\usefont{T5}{qag}{b}{n} P741.}}
	(Mức $B$) Có $6$ tô phở giống hệt nhau được xếp thành hai chồng, như ở hình dưới đây: 
	\begin{figure}[H]
		\vspace*{-5pt}
		\centering
		\captionsetup{labelformat= empty, justification=centering}
		\includegraphics[width= 1\linewidth]{1}
		\vspace*{-10pt}
	\end{figure}
	Hỏi, nếu xếp cả $6$ tô phở thành một chồng, thì chiều cao của chồng đó là bao nhiêu?
	\begin{flushright}
		\textit{Bùi Văn Biên, Pháp}
	\end{flushright}
	{\color{thachthuctoanhoc}{\usefont{T5}{qag}{b}{n} P742.}}
	(Mức $B$) Cho $10$ số hữu tỷ $a_1,\ldots,a_{10}$ khác $0$ và thoả mãn
	\vskip 0.05cm
	$i/$ $a_1+\cdots+a_8=a_1a_2\cdots a_8$. 
	\vskip 0.05cm
	$ii/$ $a_1+\cdots+a_9=a_1a_2\cdots a_9$. 
	\vskip 0.05cm
	$iii/$ $a_1+\cdots+a_{10}=a_1a_2\cdots a_{10}$. 
	\vskip 0.05cm
	Chứng minh rằng, $\dfrac{4-3a_{10}}{a_{10}}$ là bình phương của một số hữu tỷ.
	\begin{flushright}
		\textit{Lưu Bá Thắng, Hà Nội}
	\end{flushright}
	{\color{thachthuctoanhoc}{\usefont{T5}{qag}{b}{n} P743.}}
	(Mức $B$) Cho tam giác $A B C$ không cân, có đường cao $A H$ và phân giác trong $A D$. Tiếp tuyến tại $A$ của đường tròn ngoại tiếp tam giác đó cắt đường thẳng $B C$ tại $M$. Gọi $E$ là hình chiếu vuông góc của $B$ trên đường phân giác ngoài của góc $A$. Chứng minh rằng, các đường thẳng $C E, A D$ cắt nhau tại một điểm nằm trên đường tròn ngoại tiếp tam giác $A M H.$
	\begin{figure}[H]
%		\vspace*{-5pt}
		\centering
		\captionsetup{labelformat= empty, justification=centering}
		\definecolor{qqwuqq}{rgb}{0,0.39215686274509803,0}
		\definecolor{ffqqqq}{rgb}{1,0,0}
		\definecolor{qqzzff}{rgb}{0,0.6,1}
		\definecolor{qqqqff}{rgb}{0,0,1}
		\definecolor{qqqqffa}{rgb}{1,1,1}
		\begin{tikzpicture}[scale=0.5, thachthuctoanhoc]
			\draw[color=qqwuqq] (-3.07715728752538,-1.2) -- (-3.07715728752538,-0.9171572875253808) -- (-3.36,-0.9171572875253808) -- (-3.36,-1.2) -- cycle; 
			\draw [color=qqzzff] (-3.36,3.24)-- (-4.2,-1.2);
			\draw [color=qqzzff] (-4.2,-1.2)-- (3.38,-1.2);
			\draw [color=qqzzff] (3.38,-1.2)-- (-3.36,3.24);
			\draw [color=ffqqqq] (-0.41,0.3824324324324323) circle (4.107090503411634cm);
			\draw [color=qqwuqq] (-7.660881355932205,-1.2)-- (-3.36,3.24);
			\draw [color=qqzzff] (-7.660881355932205,-1.2)-- (-4.2,-1.2);
			\draw [color=qqwuqq] (-3.36,-1.2)-- (-3.36,3.24);
			\draw [dashed, color=ffqqqq] (-5.510440677966103,1.02) circle (3.0907596330758746cm);
			\draw [color=qqwuqq] (-3.36,3.24)-- (-2.419681044890228,1.02);
			\draw [color=qqwuqq] (-2.419681044890228,1.02)-- (-1.479362089780457,-1.2);
			\draw  (-5.666780525518339,2.2629236696292914)-- (-4.2,-1.2);
			\draw  (-5.666780525518339,2.2629236696292914)-- (3.38,-1.2);
			\draw  (-5.666780525518339,2.2629236696292914)-- (-1.479362089780457,-1.2);
			\draw  (-5.666780525518339,2.2629236696292914)-- (-3.36,3.24);
			\draw [fill=white] (-3.36,3.24) circle (3pt);
			\draw[color=qqqqff] (-3.36,3.71) node {$A$};
			\draw [fill=white] (-4.2,-1.2) circle (3pt);
			\draw[color=qqqqff] (-4.44,-1.8) node {$B$};
			\draw [fill=white] (3.38,-1.2) circle (3pt);
			\draw[color=qqqqff] (3.72,-1.8) node {$C$};
			\draw [fill=white] (-1.479362089780457,-1.2) circle (3pt);
			\draw[color=qqqqff] (-1.56,-1.8) node {$D$};
			\draw [fill=white] (-3.36,-1.2) circle (3pt);
			\draw[color=qqqqff] (-3.32,-1.8) node {$H$};
			\draw [fill=white] (-7.660881355932205,-1.2) circle (3pt);
			\draw[color=qqqqff] (-8.04,-1.8) node {$M$};
			\draw [fill=white] (-2.419681044890228,1.02) circle (3pt);
			%\draw[color=qqqqff] (-2.16,1.17) node {$F$};
			\draw [fill=white] (-5.666780525518339,2.2629236696292914) circle (3pt);
			\draw[color=qqqqff] (-5.86,2.85) node {$E$};
		\end{tikzpicture}
		\vspace*{-10pt}
	\end{figure}
	\begin{flushright}
		\textit{Hoàng Việt Vương, Đà Nẵng}
	\end{flushright}
	{\color{thachthuctoanhoc}{\usefont{T5}{qag}{b}{n} P744.}}
	(Mức $B$) Giải hệ phương trình 
	\begin{align*}
		\begin{cases}
			x+y=2+\sqrt{xy(xy-3)}&\\
			xy(x-y)=2(2-x).
		\end{cases}
	\end{align*}
	\begin{flushright}
		\textit{Nguyễn Viết Chương, Hà Tĩnh (st)}
	\end{flushright}
	{\color{thachthuctoanhoc}{\usefont{T5}{qag}{b}{n} P745.}}
	(Mức $B$) Xét ba số nguyên dương $x,y,z$ thoả mãn
	\begin{align*}
		\sqrt x+\sqrt y+\sqrt z=\sqrt{2023}.
	\end{align*}
	Hãy tìm giá trị lớn nhất của tích ba số đó.
	\begin{flushright}
		\textit{Nguyễn Hùng Cường, Bình Định}
	\end{flushright}
	{\color{thachthuctoanhoc}{\usefont{T5}{qag}{b}{n} P746.}}
	(Mức $B$) Tìm tất cả các số nguyên $n\ge2$ thoả mãn: với mỗi bảng ô vuông kích thước $n\times n$,  ta có thể tô mỗi ô vuông con của bảng bởi một trong hai màu đen, trắng, sao cho các điều kiện sau được đồng thời thoả mãn:  
	\vskip 0.05cm
	$i/$ Tất cả các  ô vuông con nằm ở cạnh bảng có cùng màu;
	\vskip 0.05cm
	$ii/$ Mỗi bảng con $2\times 2$ của bảng đã cho đều có ít nhất hai ô cạnh nhau cùng màu.
	\begin{flushright}
		\textit{Tô Trung Hiếu, Nghệ An (st)}
	\end{flushright}
	{\color{thachthuctoanhoc}{\usefont{T5}{qag}{b}{n} P747.}}
	(Mức $A$) Cho $a,b,c$ là các số thực dương thỏa mãn $a^2+b^2+c^2=3abc$. Chứng minh rằng
	\begin{align*}
		\dfrac{1}{a^2}+\dfrac{1}{b^2}+\dfrac{1}{c^2}-\dfrac{9}{4(ab+bc+ca)}\le \dfrac94.
	\end{align*}
	\begin{flushright}
		\textit{Hoàng Ngọc Minh, Hà Nội}
	\end{flushright}
	{\color{thachthuctoanhoc}{\usefont{T5}{qag}{b}{n} P748.}}
	(Mức $A$) Cho tam giác $ABC$ nội tiếp đường tròn $(O)$, có $M$ là trung điểm $BC$. Các tiếp tuyến tại $B,C$ của $(O)$ cắt nhau tại $D$. Gọi $N$ là hình chiếu vuông góc của $O$ trên $AD$. Trên đường trung trực của $BC$ lấy điểm $K$, không nằm trên $(O)$ và khác $M$. Đường tròn ngoại tiếp tam giác $KBC$ cắt đường thẳng $AK$ tại điểm thứ hai $F$. Đường thẳng $NF$ và $AM$ cắt nhau tại $E$. Chứng minh rằng tam giác $AEF$ cân.
	\begin{figure}[H]
		\vspace*{-5pt}
		\centering
		\captionsetup{labelformat= empty, justification=centering}
		\definecolor{qqwuqq}{rgb}{0,0.39215686274509803,0}
		\definecolor{qqzzff}{rgb}{0,0.6,1}
		\definecolor{ffqqqq}{rgb}{1,0,0}
		\definecolor{qqqqff}{rgb}{0,0,1}
		\definecolor{qqqqffa}{rgb}{1,1,1}
		\begin{tikzpicture}[thachthuctoanhoc]
			\draw[color=qqwuqq] (-1.7477321892935256,1.132220777334539) -- (-1.7883916969635338,1.3145605332935437) -- (-1.9707314529225384,1.2739010256235357) -- (-1.9300719452525303,1.091561269664531) -- cycle; 
			\draw  (-2.405870046259361,3.2253035256569276)-- (-3,0);
			\draw  (-3,0)-- (1,0);
			\draw  (1,0)-- (-2.405870046259361,3.2253035256569276);
			\draw [color=ffqqqq] (-1,1.2989557963971259) circle (2.3848031702833867cm);
			\draw  (-2.405870046259361,3.2253035256569276)-- (-1,-3.079396551518288);
			\draw [color=qqzzff] (-1,0.18626609472755565) circle (2.0086550370944867cm);
			\draw  (-3,0)-- (-1,-3.079396551518288);
			\draw  (1,0)-- (-1,-3.079396551518288);
			\draw  (-2.405870046259361,3.2253035256569276)-- (0.9154399399581276,0.7910676278425582);
			\draw  (-2.405870046259361,3.2253035256569276)-- (-1,0);
			\draw  (-1.9300719452525303,1.091561269664531)-- (-1,1.2989557963971259);
			\draw  (-1.4538784873938124,1.0412739708809693)-- (0.9154399399581276,0.7910676278425582);
			\draw [fill=white] (-2.405870046259361,3.2253035256569276) circle (1.5pt);
			\draw[color=qqqqff] (-2.4983402610906977,3.601789400327366) node {$A$};
			\draw [fill=white] (-3,0) circle (1.5pt);
			\draw[color=qqqqff] (-3.264522041121767,-0.11022922361625731) node {$B$};
			\draw [fill=white] (1,0) circle (1.5pt);
			\draw[color=qqqqff] (1.1344181787117853,-0.13664928499663898) node {$C$};
			\draw [fill=white] (-1,-3.079396551518288) circle (1.5pt);
			\draw[color=qqqqff] (-0.9923967624089413,-3.367426558571866) node {$D$};
			\draw [fill=white] (-1,1.2989557963971259) circle (1.5pt);
			\draw[color=qqqqff] (-0.8074563327462693,1.4221343364458792) node {$O$};
			\draw [fill=white] (-1,2.1949211318220425) circle (1.5pt);
			\draw[color=qqqqff] (-0.8999265475776053,2.3996766075200004) node {$K$};
			\draw [fill=white] (-1,0) circle (1.5pt);
			\draw[color=qqqqff] (-1.0452368851697047,-0.202343925430644814) node {$M$};
			\draw [fill=white] (0.9154399399581276,0.7910676278425582) circle (1.5pt);
			\draw[color=qqqqff] (1.0155279025000676,0.9994133543597725) node {$F$};
			\draw [fill=white] (-1.9300719452525303,1.091561269664531) circle (1.5pt);
			\draw[color=qqqqff] (-2.115249371075163,0.8673130474578641) node {$N$};
			\draw [fill=white] (-1.4538784873938124,1.0412739708809693) circle (1.5pt);
			\draw[color=qqqqff] (-1.705738419679247,0.7484227712461466) node {$E$};
		\end{tikzpicture}
		\vspace*{-10pt}
	\end{figure}
	\begin{flushright}
		\textit{Trần Tú Anh, Hải Phòng}
	\end{flushright}
	{\color{thachthuctoanhoc}{\usefont{T5}{qag}{b}{n} P749.}}
	(Mức $A$) Cho dãy số $(a_n)$ với $a_1=a$ là một số nguyên và 
	\begin{align*}
		a_{n+1}=a_n^3+2a_n^2-3a_n-8\quad\text{với mọi $n\ge1$.}
	\end{align*}
	Tìm tất cả các số nguyên $a$  sao cho 
	\begin{align*}
		a_{2023}\equiv508\pmod{2024}.
	\end{align*}
	\begin{flushright}
		\textit{Nguyễn Tuấn Ngọc, Tiền Giang}
	\end{flushright}
	{\color{thachthuctoanhoc}{\usefont{T5}{qag}{b}{n} P750.}}
	(Mức $A$) Một cuộc thi Olympic Toán học diễn ra hàng năm trong $2$ ngày thi, mỗi ngày thi có $3$ bài toán, mỗi bài toán có điểm tối đa là $7$. Năm nay, cuộc thi này có sự tham gia của $40$ đội tuyển (mỗi đội tuyển có ít nhất $1$ học sinh). Thống kê điểm sau khi hoàn tất khâu chấm thi,  ban giám khảo nhận thấy rằng số điểm các em đạt được là những số nguyên từ $14$ đến $42$. 
	Chứng minh rằng ta có thể chọn ra một nhóm có ít nhất $9$ em học sinh, sao cho với mỗi em trong nhóm, số học sinh cùng điểm với em đó lớn hơn số học sinh cùng đội tuyển với em đó.
	\begin{flushright}
		\textit{Trần Anh Tùng, Hà Nội}
	\end{flushright}
\end{multicols}
\newpage
\centerline{{\large{\textbf{\color{thachthuctoanhoc}\color{thachthuctoanhoc}GIẢI BÀI KỲ TRƯỚC}}}}
\vspace*{-5pt}
\begin{multicols}{2}
	\setlength{\abovedisplayskip}{5pt}
	\setlength{\belowdisplayskip}{5pt}
	{\color{thachthuctoanhoc}{\usefont{T5}{qag}{b}{n} P720.}}
	(Mức $A$) Một thành phố có $1332$ căn nhà. Mỗi dịp Noel, Ông già Noel sẽ đến thăm các căn nhà đó theo thứ tự tùy ý. Chứng minh rằng, có thể tìm được $12$ căn nhà trong thành phố đó, sao cho trong ba năm liên tiếp, có ít nhất hai năm mà Ông già Noel đến thăm $12$ căn nhà đó theo cùng một thứ tự.
	\vskip 0.05cm
	\textbf{Lời giải} (\textit{dựa theo Đáp án của bài toán})\textbf{.}
	\vskip 0.05cm
	Trước hết, ta nhắc lại kết quả nổi tiếng sau:
	\vskip 0.05cm
	\textbf{Định lý Erdos -- Szekeres.} Cho các số nguyên dương $p, q > 1$. Khi đó, mỗi dãy $(p - 1)(q - 1) + 1$ số thực đôi một phân biệt sẽ chứa một dãy con tăng có $p$ số hạng, hoặc chứa một dãy con giảm có $q$ số hạng.
	\vskip 0.05cm
	\textit{Chứng minh.}
	\vskip 0.1cm
	Đặt $N = (p - 1)(q - 1) + 1$.
	\vskip 0.05cm
	Xét dãy số thực $x_1,x_2,\ldots,x_N$ tùy ý, thỏa mãn $x_i \ne x_j$, với mọi $i, j \in \{1; 2; \ldots; N\}$ và $i \ne j$.
	\vskip 0.05cm
	Với mỗi $i \in \{1; 2; \ldots; N\}$, kí hiệu $d_i$ là số các số hạng của dãy con tăng có nhiều số hạng nhất và có $x_i$ là số hạng có giá trị lớn nhất; kí hiệu $n_i$ là số các số hạng của dãy con giảm có nhiều số hạng nhất và có  $x_i$ là số hạng có giá trị bé nhất.
	\vskip 0.05cm
	Xét các cặp số $\left(d_i, n_i\right),  i = 1, 2, \ldots, N$.
	\vskip 0.05cm
	Dễ thấy, với $1 \le i < j \le N$, ta có  $d_i < d_j$ nếu  $x_i < x_j$, và $n_i < n_j$  nếu $x_i > x_j$. Vì vậy, $N$ cặp $\left(d_i, n_i\right), i = 1, 2, \ldots, N$, đôi một khác nhau.     \hfill ($1$)
	\vskip 0.05cm
	Nhận thấy, nếu với mọi $i \in {1; 2; \ldots; N}$, $1 \le d_i \le p-1$  và $1 \le n_i \le q-1$, thì trong $N$ cặp $\left(d_i, n_i\right), i = 1, 2, \ldots, N$, chỉ có tối đa $(p - 1)(q - 1)$ cặp đôi một khác nhau, mâu thuẫn với ($1$) (do $(p - 1)(q - 1) < N$). Vì vậy, phải tồn tại $i \in {\1; 2; \ldots; N\}$ sao cho $d_i \ge p$, hoặc tồn tại $j \in \{1; 2; \ldots; N\}$ sao cho $n_j \ge q$. Từ đây, hiển nhiên ta có điều phải chứng minh theo yêu cầu của định lý.
	\vskip 0.05cm
	\textit{Trở lại bài toán.}
	\vskip 0.05cm
	Xét ba năm liên tiếp tùy ý.
	\vskip 0.05cm
	Ở năm thứ nhất, theo chân Ông già Noel, ta lần lượt đánh số các căn nhà mà Ông tới thăm, bởi$ 1, 2, \ldots, 1332$. Ta sẽ gọi căn nhà được đánh số $i$ là căn nhà $i$.
	\vskip 0.05cm
	Khi đó, ở năm thứ hai, liệt kê các căn nhà theo thứ tự mà Ông già Noel lần lượt tới thăm, ta sẽ thu được một hoán vị của $1, 2, \ldots, 1332$:
	\begin{align*}
		{a_1},\,\,{a_2},\,\, \ldots ,\,\,{a_{1332}}. \tag{$2$}
	\end{align*}
	Dễ thấy, nếu trong dãy ($2$) tồn tại một dãy con tăng có $12$ số hạng thì $12$ số hạng của dãy con đó sẽ là $12$ số hạng nằm theo cùng một thứ tự trong dãy $1, 2, \ldots, 1332$ và trong dãy ($2$). Vì thế, ta có $12$ căn nhà, mà Ông già Noel tới thăm ở năm thứ nhất và năm thứ hai theo cùng một thứ tự.
	\vskip 0.05cm
	Xét trường hợp ngược lại, trong dãy ($2$) không tồn tại một dãy con tăng có $12$ số hạng.
	\vskip 0.05cm
	Khi đó, do
	\begin{align*}
		1332 = (12 - 1)(122 - 1) + 1,
	\end{align*}
	nên theo định lý Erdos -- Szekeres, trong dãy ($2$) phải tồn tại một dãy con giảm có $122$ số hạng. Giả sử dãy con đó là
	\begin{align*}
		{a_{{i_1}}},\,\,{a_{{i_2}}},\,\, \ldots ,\,\,{a_{{i_{122}}}}. \tag{$3$}
	\end{align*}
	Xét năm thứ ba. Giả sử trong năm này, Ông già Noel tới thăm các căn nhà thuộc dãy ($3$) theo thứ tự:
	\begin{align*}
		{a_{{j_1}}},\,\,{a_{{j_2}}},\,\, \ldots ,\,\,{a_{{j_{122}}}}. \tag{$4$}
	\end{align*}
	Dễ thấy, nếu trong dãy ($4$) tồn tại một dãy con tăng có $12$ số hạng thì $12$ số hạng của dãy con đó sẽ là $12$ số hạng nằm trong dãy $1, 2, \ldots, 1332$ và dãy ($4$) theo cùng một thứ tự. Vì thế, ta có $12$ căn nhà, mà Ông già Noel tới thăm ở năm thứ nhất và năm thứ ba theo cùng một thứ tự.
	\vskip 0.05cm
	Xét trường hợp ngược lại, trong dãy ($4$) không tồn tại một dãy con tăng có $12$ số hạng.
	\vskip 0.05cm
	Khi đó, do
	\begin{align*}
		122 = (12 - 1)(12 - 1) + 1,
	\end{align*}
	nên theo định lý Erdos -- Szekeres, trong dãy ($4$) phải tồn tại một dãy con giảm có $12$ số hạng; kí hiệu dãy con này là ($5$).
	\vskip 0.05cm
	Do tất cả $12$ số hạng của dãy ($5$) đều là số hạng của dãy ($3$), và do cả hai dãy ($3$), ($5$) cùng là dãy giảm, nên tất cả $12$ số hạng của dãy ($5$) nằm trong dãy ($3$) và dãy ($4$) theo cùng một thứ tự. Vì thế, ta có $12$ căn nhà, mà Ông già Noel tới thăm ở năm thứ hai và năm thứ ba theo cùng một thứ tự.
	\vskip 0.05cm
	Kết quả xét các trường hợp có thể xảy ra trên đây cho ta điều phải chứng minh theo yêu cầu đề bài.
	\vskip 0.05cm
	\textbf{Bình luận và Nhận xét}
	\vskip 0.05cm
	Cho tới thời điểm bản thảo vào Nhà in, Tạp chí vẫn chưa nhận được lời giải nào từ bạn đọc.
	\begin{flushright}
		\textbf{Nguyễn Khắc Minh}
	\end{flushright}
	{\color{thachthuctoanhoc}{\usefont{T5}{qag}{b}{n} P721.}}
	(Mức $B$)
	Trên mỗi cạnh của một hình vuông, bạn An viết một số nguyên dương. Sau đó, tại mỗi đỉnh của hình vuông đó, bạn An viết một số bằng tích của hai số đã được viết ở hai cạnh đi qua đỉnh đó. Biết rằng, tổng các số ở các đỉnh của hình vuông bằng $1333$. Hỏi, tổng các số được viết ở các cạnh của hình vuông đó có thể bằng bao nhiêu?
	\vskip 0.05cm
	\textbf{Lời giải} (\textit{dựa theo ý giải của một bạn học sinh cấp THCS})\textbf{.}
	\vskip 0.05cm
	Giả sử các số nguyên dương được viết ở các cạnh của hình vuông, tính theo chiều kim đồng hồ, lần lượt là $a, b, c, d$.
	\vskip 0.05cm
	Khi đó, các số được viết ở bốn đỉnh của hình vuông đó sẽ là $ab$, $bc$, $cd$ và $da$.
	\vskip 0.05cm
	Theo giả thiết của bài ra, ta có:
	\begin{align*}
		31 \cdot 43\,\, = \,\,1333\,\, = \,\,ab\,\, + \,\,bc\,\, + \,\,cd\,\, + \,\,da\,\, = \,\,\left( {a\,\, + \,\,c} \right)\left( {b\,\, + \,\,d} \right). \tag$(*)$
	\end{align*}
	Do $a + c, b + d$ là các số nguyên dương lớn hơn $1$ (vì $a,b,c,d \in \mathbb{N^*}$) và $31$, $43$ là các số nguyên tố, nên
	\begin{align*}
		\left( *  \right)\,\, \Leftrightarrow \,\,\left( {a\,\, + \,\,c,\,\,b\,\, + \,\,d} \right)\,\, \in \,\,\left\{ {\left( {31,\,\,43} \right);\,\,\left( {43,\,\,31} \right)} \right\}.
	\end{align*}
	Vì vậy, $a + b + c + d = 31 + 43 = 74$.
	\vskip 0.05cm
	Vậy, tổng các số được viết ở các cạnh của hình vuông bằng $74$.
	\vskip 0.05cm
	\textbf{Bình luận và Nhận xét}
	\vskip 0.05cm	
	Tuy bài đã ra là một bài toán đơn giản, nhưng rất tiếc, trong số các lời giải Tạp chí đã nhận được từ bạn đọc, có một số lời giải thiếu chặt chẽ, thiếu chính xác, do người giải bài đã mắc một trong các lỗi chuyên môn sau:
	\vskip 0.05cm
	-- Bỏ sót phân tích $1333 = 1 \cdot 1333$, khi xét các phân tích số $1333$ thành tích của hai số nguyên dương;
	\vskip 0.05cm
	-- Thiếu khẳng định $1333$ chỉ có đúng hai cách phân tích thành tích của hai số nguyên dương, là
	\begin{align*}
		1333 = 1 \cdot 1333 \text{ và } 1333 = 31 \cdot 43.
	\end{align*}  
	\begin{flushright}
		\textbf{Hà Thanh}
	\end{flushright}
	{\color{thachthuctoanhoc}{\usefont{T5}{qag}{b}{n} P722.}}
	(Mức $B$)
	Cho $a, b, c$ là các số thực khác $0$, thỏa mãn:
	\begin{align*}
		\begin{cases}
			a\,\, + \,\,b\,\, + \,\,c\,\, = \,\,\frac{1}{a}\,\, + \,\,\frac{1}{b}\,\, + \,\,\frac{1}{c}\\
			{a^3}\, + \,\,{b^3}\, + \,\,{c^3}\, = \,\,\frac{1}{{{a^3}}}\,\, + \,\,\frac{1}{{{b^3}}}\,\, + \,\,\frac{1}{{{c^3}}}.
		\end{cases}
	\end{align*}
	Chứng minh rằng
	\begin{align*}
		{a^{2023}}\, + \,\,{b^{2023}}\, + \,\,{c^{2023}}\, = \,\,\frac{1}{{{a^{2023}}}}\,\, + \,\,\frac{1}{{{b^{2023}}}}\,\, + \,\,\frac{1}{{{c^{2023}}}}.
	\end{align*}
	\textbf{Lời giải} (\textit{của người chấm bài})\textbf{.}
	\vskip 0.05cm
	Đặt ,   và  .
	Theo giả thiết của bài ra, ta có:
	(1)
	và
	
	(2)
	Từ (1) suy ra x + y = z. Do đó, từ (2) ta được:
	
	Vì vậy, x = 0, hoặc y = 0, hoặc z = 0. Điều này cho thấy, trong ba số a, b, c, có ít nhất một số bằng nghịch đảo của nó.
	Do vai trò của a, b, c trong bài toán hoàn toàn như nhau, nên không mất tính tổng quát, giả sử
	.                                                                          (3)
	Khi đó, từ giả thiết
	,
	suy ra
	.
	Do đó, b + c = 0 hoặc bc = 1.
	- Nếu b + c = 0 thì b =  c. Từ đây và (3), suy ra
	
	- Nếu bc = 1 thì
	
	Vì vậy, ta có điều phải chứng minh theo yêu cầu đề bài.
	Bình luận và Nhận xét
	1. Ở bài đã ra, nếu thay giả thiết
	,
	bởi giả thiết “ ”, ta có thể dễ dàng chứng minh được rằng
	,
	với mọi số nguyên dương n > 1.
	2. Trong số các lời giải Tạp chí đã nhận được từ bạn đọc, rất tiếc, có ba lời giải sai, do người giải bài đã mắc ít nhất một trong các lỗi chuyên môn sau:
	- Chưa xét hết các trường hợp có thể xảy ra đối với a, b, c;
	- Nhầm lẫn giữa “hoặc” và “đồng thời”.
	Hà Thanh
	P723. (Mức B) Cho số nguyên dương k và cho A là số tự nhiên gồm k chữ số 9. Gọi m, n, p tương ứng là tổng các chữ số của A,     Chứng minh rằng p = 2m = 2n.
	Lời giải (dựa theo lời giải của bạn Nguyễn Chánh Thiện, lớp 9/12, trường THCS Lê Quý Đôn, Quận 3, Tp. Hồ Chí Minh).
	Theo giả thiết của bài ra,                                                                                                     (1)
	Vì vậy, m = 9k.                                                                                                                                        (2)
	Từ (1), ta có:
	
	
	Vì vậy
	n = 9(k - 1) + 8 + 1 = 9k,                                                          (3)
	p = 9(k - 1) + 7 + 2 + 9k = 18k.                                                     (4)
	Từ (2), (3) và (4), suy ra p = 2m = 2n, là điều phải chứng minh theo yêu cầu đề bài.
	Bình luận và Nhận xét
	Tất cả lời giải Tạp chí nhận được từ bạn đọc đều là lời giải đúng và hoàn chỉnh.
	Lưu Thị Thanh Hà
	P724. (Mức B) Cho tam giác ABC. Trên các cạnh BC, CA, AB, tương ứng, lấy các điểm D, E, F, sao cho AEDF là hình bình hành (D, E, F không trùng với các đỉnh của tam giác). Gọi Y là điểm đối xứng với B qua DF; Z là điểm đối xứng với C qua DE. Chứng minh rằng, đường tròn ngoại tiếp tam giác AYZ đi qua trực tâm của tam giác ABC.
	Lời giải (dựa theo Đáp án của BBT Tạp chí).
	Gọi H là trực tâm của tam giác ABC.
	Xét hai trường hợp sau:
	• Trường hợp 1: D không là trung điểm của BC.
	
	Gọi d là đường thẳng vuông góc với BC tại D. Gọi   tương ứng là điểm đối xứng với A, B, C qua d. Gọi K là giao điểm của   và  
	Do Z đối xứng với C qua DE (giả thiết) nên DE là đường trung trực của CZ. Từ đây và định nghĩa điểm   suy ra   Do đó
	(1)
	suy ra,   (vì cùng vuông góc với CZ). Mà   (do tứ giác AEDF là hình bình hành), nên                                                                                                                                           (2)
	Xét các điểm Y, B,   một cách hoàn toàn tương tự, ta cũng chứng minh được
	(3)
	và                                                                                                                                            (4)
	Từ (1) và (2) suy ra   mà CH cũng vuông góc BA (do H là trực tâm tam giác ABC), nên ba điểm C, H, Z thẳng hàng. Do đó, theo (1), ta có   suy ra                           (5)
	Một cách hoàn toàn tương tự, từ (3) và (4) suy ra                                                            (6)
	Do cùng vuông góc với d nên                                                                                                (7)
	Từ (2) và (7) suy ra tứ giác   là một hình bình hành. Do đó,                                     (8)
	Do phép đối xứng trục bảo toàn khoảng cách, nên   Kết hợp với (8), ta được   Mà   (vì cùng vuông góc với d), nên   là một hình thang cân. Suy ra
	(9)
	Do phép đối xứng trục bảo toàn góc, nên   Kết hợp với (9), ta được
	
	Do đó,   Kết hợp với (4), suy ra ba điểm   Y, K thẳng hàng. Vì thế, từ (6) suy ra
	(10)
	Do   (vì cùng vuông góc với d) và AH  BC (vì H là trực tâm tam giác ABC), nên
	(11)
	Từ (5), (10) và (11) suy ra, các điểm A, Y, Z, H, K cùng nằm trên đường tròn đường kính HK.
	• Trường hợp 2: D là trung điểm của BC.
	Trong trường hợp này, DE và DF là các đường trung bình của tam giác ABC. Do đó, Y là chân đường cao kẻ từ B, và Z là chân đường cao kẻ từ C, của tam giác ABC. Vì thế, các điểm A, Y, Z, H cùng nằm trên đường tròn đường kính AH.
	Kết quả xét hai trường hợp trên đây cho ta điều phải chứng minh theo yêu cầu đề bài.
	Bình luận và Nhận xét
	1. Theo đánh giá chủ quan của người chấm bài, bài đã ra là một bài toán hay, và không quá khó đối với học sinh khá, giỏi toán cấp THCS.
	2. Với các giả thiết của bài đã ra, có thể chứng minh được rằng, tỉ số   không thay đổi, khi điểm D di động trên cạnh BC sao cho D không trùng với B, C và trung điểm của BC. Mời các độc giả có quan tâm cùng chứng minh điều vừa nêu.
	3. Hầu hết các lời giải Tạp chí đã nhận được từ bạn đọc đều có nhược điểm: lời giải chỉ đúng cho thế hình mà người giải bài đã vẽ. Tất cả những lời giải như vậy, hiển nhiên, không là lời giải hoàn chỉnh.
	Hạ Vũ Anh
	P725. (Mức B) Cho các số dương a, b, c. Chứng minh rằng
	
	Lời giải (dựa theo lời giải của các bạn: Lê Nguyễn Hoàng Nhật Đình, lớp 9C, trường THCS Nguyễn Thái Bình, Tỉnh Cà Mau, và Nguyễn Chánh Thiện, lớp 9/12, trường THCS Lê Quý Đôn, Quận 3, Tp. Hồ Chí Minh).
	Kí hiệu P là biểu thức ở vế trái của bất đẳng thức cần chứng minh theo yêu cầu đề bài.
	Theo bất đẳng thức Cauchy - Schwarz, ta có:
	(1)
	Tiếp theo, ta sẽ chứng minh
	(2)
	Thật vậy, ta có:
	
	(3)
	Vì a, b, c > 0 (giả thiết), nên (3) là bất đẳng thức đúng. Vì vậy, (2) được chứng minh.
	Từ (1) và (2), hiển nhiên suy ra bất đẳng thức cần chứng minh theo yêu cầu đề bài.
	Bình luận và Nhận xét
	1. Dễ thấy, dấu “=” ở bất đẳng thức của đề bài xảy ra khi và chỉ khi a = b = c.
	2. Trong số các lời giải Tạp chí nhận được từ bạn đọc, rất tiếc, có một lời giải sai, do người giải bài đã ngộ nhận rằng, có thể nhân hai bất đẳng thức trái chiều, A  B > 0 và 0 < C \le D, để suy ra bất đẳng thức AC \le BD.
	3. Trong Đề thi chọn Đội tuyển học sinh Việt Nam tham dự Olympic Toán học Quốc tế năm 1982 (IMO 1982) có bài toán sau:
	Bài toán thi chọn Đội tuyển học sinh VN tham dự IMO 1982. Cho các số thực dương a, b, c. Chứng minh rằng
	
	Với a, b, c là các số thực dương, theo bất đẳng thức hoán vị, ta có:
	
	Vì thế, có thể nói, bất đẳng thức ở bài đã ra là một phương án làm chặt bất đẳng thức trong Đề thi chọn Đội tuyển học sinh VN tham dự IMO 1982.
	4. Có thể dễ dàng chứng minh được rằng, k = 3 là số thực dương lớn nhất, sao cho
	
	với mọi a, b, c > 0.
	Trần Nam Dũng
	P726. (Mức B) Cho bảng ô vuông kích thước 2023  2023, mà ở mỗi ô vuông con được đặt ít nhất một viên bi. Cho phép thay đổi số bi trong bảng, theo qui tắc: Mỗi lần, thêm bi vào mỗi ô vuông con của một cột tùy ý, sao cho số bi ở mỗi ô của cột đó tăng lên gấp đôi; hoặc bớt một viên bi ở mỗi ô vuông con của một hàng tùy ý, mà ở tất cả các ô của hàng đó đều đang có bi. Chứng minh rằng, ta có thể thực hiện một số lần phép thay đổi bi nói trên, để trong bảng không còn viên bi nào.
	Lời giải (của người chấm bài).
	Để thuận tiện cho việc diễn đạt, ta qui ước:
	- Gọi phép thay đổi số bi ở một cột, theo qui tắc của đề bài, là phép “nhân đôi”;
	- Gọi phép thay đổi số bi ở một hàng, theo qui tắc của đề bài, là phép “giảm bớt”.
	Ta có Nhận xét sau:
	Nhận xét. Nhờ việc thực hiện một số hữu hạn lần liên tiếp phép thay đổi bi đã cho trong đề bài, có thể làm cho một hàng tùy ý của bảng không còn viên bi nào.
	Chứng minh.
	Xét một hàng tùy ý của bảng; gọi là hàng H. Có thể xảy ra các trường hợp sau:
	• Trường hợp 1: Tất cả các ô của hàng H đều có số bi như nhau.
	Giả sử tất cả các ô đều có k viên bi.
	Khi đó, bằng cách thực hiện liên tiếp k lần phép “giảm bớt” đối với hàng H, ta sẽ làm cho hàng này không còn viên bi nào.
	• Trường hợp 2: Tất cả các ô của hàng H không có số bi như nhau.
	Có thể xảy ra hai trường hợp nhỏ sau:
	♦ Trường hợp 2.1: Tồn tại ô có đúng 1 viên bi.
	Xét phương án thực hiện phép thay đổi bi, gồm hai bước liên tiếp sau:
	- Bước 1: Lần lượt, thực hiện phép “nhân đôi” đối với từng cột chứa ô có 1 viên bi của hàng H;
	- Bước 2: Thực hiện phép “giảm bớt” đối với hàng H.
	Ta gọi mỗi phương án nêu trên là một “tổ hợp tăng - giảm”.
	Dễ thấy:
	- Ô có đúng 1 viên bi, tại thời điểm ngay trước khi thực hiện “tổ hợp tăng - giảm”, vẫn sẽ có đúng 1 viên bi tại thời điểm ngay sau khi thực hiện “tổ hợp tăng - giảm” đó;
	- Ô có nhiều bi nhất trong hàng, tại thời điểm ngay trước khi thực hiện “tổ hợp tăng - giảm”, vẫn sẽ là ô có nhiều bi nhất trong hàng tại thời điểm ngay sau khi thực hiện “tổ hợp tăng - giảm” đó, nhưng số bi trong ô này bị giảm đi 1.
	Từ đó suy ra, nếu gọi M là số viên bi của ô có nhiều bi nhất trong hàng H tại thời điểm ban đầu (tức, thời điểm bắt đầu “dọn” bi trong hàng đó), thì bằng cách thực hiện liên tiếp M - 1 lần “tổ hợp tăng - giảm”, ta sẽ làm cho mỗi ô trong hàng H đều có đúng 1 viên bi. Lúc này, bằng cách thực hiện phép “giảm bớt”, ta sẽ làm cho trong hàng H không còn viên bi nào.
	♦ Trường hợp 2.2: Mỗi ô đều có ít nhất 2 viên bi.
	Gọi m là số viên bi của ô có ít bi nhất trong hàng H, tại thời điểm ban đầu; ta có m  2.
	Dễ thấy, bằng cách thực hiện liên tiếp m - 1 lần phép “giảm bớt” đối với hàng H, ta sẽ làm cho hàng đó có những ô có đúng 1 viên bi. Lúc này, trạng thái bi ở hàng H là trường hợp 2.1. Vì thế, ta có thể làm cho trong hàng đó không còn viên bi nào.
	Nhận xét được chứng minh.
	Dễ thấy, khi làm cho một hàng nào đó không còn bi, theo cách đã trình bày trong chứng minh của Nhận xét trên, ta không làm xuất hiện bi ở những ô không có bi của các hàng khác. Vì vậy, theo Nhận xét trên, bằng cách lần lượt “dọn” bi ở từng hàng, ta sẽ làm cho trong bảng không còn viên bi nào.
	Bình luận và Nhận xét
	Tât cả các lời giải Tạp chí đã nhận được từ bạn đọc, rất tiếc, đều là lời giải không hoàn chỉnh, do các lập luận trong lời giải không chặt chẽ, thiếu chính xác.
	Nguyễn Khắc Minh
	P728. (Mức A) Tìm tất cả các cặp số hữu tỉ (a, b) sao cho tồn tại duy nhất hàm số   thỏa mãn:
	
	với mọi  .
	Lời giải (của người chấm bài).
	Giả sử   là hàm số thỏa mãn
	(1)
	với mọi  ; trong đó, a, b là các hằng số hữu tỉ.
	Do   xác định trên   và thỏa mãn (1) với mọi  , nên phải có
	
	với mọi  .
	Suy ra,   với mọi  .                                                                                                        (2)
	Trong (1), cho y = 0, ta được:
	(3)
	với mọi  .
	Với mỗi  , đặt   ta có  
	Do   là hàm số xác định trên   và lấy giá trị trong   (theo (2)), nên g là một hàm số xác định trên   và lấy giá trị trong  .
	Với x, y tùy ý thuộc  , theo (1) và (3), ta có:
	
	Như vậy, ta đã chứng minh được:
	(4)
	với mọi  .
	Trong (4), cho x = 0, với lưu ý   ta được:
	(5)
	với mọi  . Suy ra
	(6)
	với mọi  .
	Trong (5), thay y bởi   ta được:
	(7)
	với mọi  .
	Trong (4), thay đồng thời x bởi   và y bởi by, ta được:
	(8)
	với mọi  .
	Từ (6), (7) và (8), suy ra
	
	với mọi  .
	Do đó
	(9)
	với mọi  .
	Trong (4), thay y bởi   và sử dụng (5), ta được:
	(10)
	với mọi  .
	Trong (4), thay đồng thời x bởi   và y bởi by, ta được:
	(11)
	với mọi  .
	Từ (11), (9) và (10), suy ra
	(12)
	với mọi  .
	Trong (12), thay y bởi   ta được:
	
	với mọi  .
	Sử dụng hệ thức vừa thu được ở trên, bằng phương pháp qui nạp theo   dễ dàng chứng minh được
	
	với mọi   và với mọi  .
	Từ đó, dễ dàng chứng minh được
	
	với mọi   và với mọi  
	Vì vậy,   với   là một hằng số hữu tỉ.
	Suy ra
	(13)              
	với   là các hằng số hữu tỉ.
	Ngược lại, thay (13) vào (1), ta được:
	
	với mọi  .
	Điều này tương đương với
	(I)
	Như vậy,   là hàm số thỏa mãn (1) khi và chỉ khi   là hàm số được xác định bởi (13); trong đó,   là nghiệm của hệ phương trình (I) (ẩn  ).
	Do đó, (a, b) là cặp số hữu tỉ thỏa mãn yêu cầu đề bài khi và chỉ khi a, b là các số hữu tỉ sao cho hệ phương trình (I) (ẩn  ) có nghiệm duy nhất.
	Do phương trình thứ hai của hệ (I) không có nghiệm   nên hệ đó có nghiệm chỉ khi phương trình thứ nhất của nó có nghiệm   Điều vừa nêu xảy ra khi và chỉ khi a = 1.
	Với a = 1, dễ thấy, hệ (I) sẽ có vô số nghiệm, nếu phương trình thứ hai của nó có nghiệm  
	Vì vậy, hệ phương trình (I) có nghiệm duy nhất khi và chỉ khi a = 1 và phương trình thứ hai của hệ đó có nghiệm duy nhất, khác 1. Điều vừa nêu xảy ra khi và chỉ khi
	(II)
	Giải hệ (II), ta được b = 2.
	Vậy,   là cặp số hữu tỉ duy nhất thỏa mãn yêu cầu đề bài.
	Bình luận và Nhận xét
	1. Từ Lời giải trên, hiển nhiên thấy, hàm số   xác định trên  , là nghiệm hàm duy nhất của phương trình hàm đã nêu trong đề bài, ứng với  
	2. Tạp chí chỉ nhận được duy nhất lời giải cho bài đã ra, và rất tiếc, lời giải đó lại là một lời giải sai, do người giải bài dã biến đổi sai một số phương trình hàm được đề cập trong lời giải.
	Trần Nam Dũng - Nguyễn Khắc Minh
	P729. (Mức A) Cho tam giác nhọn ABC, có M là trung điểm BC. Trên cạnh AC, lấy một điểm E tùy ý, khác A và C. Gọi N là giao điểm của các đường thẳng BE và AM. Đường thẳng CN cắt đường tròn (ACM) tại điểm thứ hai D. Chứng minh rằng, đường thẳng đi qua D, vuông góc với CD, cũng đi qua tâm đường tròn ngoại tiếp tam giác ABE.
	Lời giải (dựa theo Đáp án của BBT Tạp chí).
	
	Đường thẳng CN cắt đường tròn (ABE) tại hai điểm X, Y, cắt AB tại F, cắt đường tròn (AEF) tại điểm thứ hai G, và cắt đường tròn (ABC) tại điểm thứ hai H.
	Áp dụng định lý Ceva cho tam giác ABC với ba đường đồng qui AM, BE, CF, ta được:
	(do M là trung điểm BC).
	Suy ra,   Do đó, theo định lí Thales đảo,                                                                (1)
	Xét phương tích của F đối với các đường tròn (ABE) và (ABC), ta được:
	
	Suy ra,                                                                                                                                  (2)
	Ta có:
	(3)
	Xét phương tích của C đối với các đường tròn (AEF) và (ABE), ta được:
	
	Suy ra
	
	Do đó
	(do (2)).                                                      (4)
	Từ (3) và (4), suy ra  
	Vì vậy, XY và GH có cùng trung điểm.                                                                                                  (5)
	Do A, B, C, H cùng thuộc một đường tròn, nên
	(6)
	Do (1) và do A, E, F, G cùng thuộc một đường tròn, nên
	(7)
	Từ (6) và (7), suy ra AHB  AGC.
	Do đó,   và
	
	Suy ra, AHG  ABC.
	Vì vậy, gọi   là trung điểm HG, ta có   Do đó
	
	Suy ra,   thuộc đường tròn (AMC). Mà C, D,   thẳng hàng và D thuộc đường tròn (AMC), nên   Điều này cho thấy, D là trung điểm của HG; vì thế, theo (5), D là trung điểm của XY. Do đó, đường thẳng vuông góc với CD tại D là đường trung trực của dây cung XY của đường tròn (ABE). Vì vậy, nó đi qua tâm của đường tròn đó. Ta có điều phải chứng minh theo yêu cầu đề bài.
	Bình luận và Nhận xét
	Lời giải của bạn Nguyễn Gia Khánh (lớp 12 Toán, trường THPT chuyên Hưng Yên, Tỉnh Hưng Yên) là lời giải duy nhất Tạp chí nhận được từ bạn đọc, và là một lời giải đúng.
	Hạ Vũ Anh
	P730. (Mức A) Với mỗi số nguyên m > 1, kí hiệu   tương ứng là tổng tất cả các ước nguyên tố của m, và số các ước nguyên tố của m. Cho n là một số nguyên lẻ, lớn hơn 1, và không chia hết cho 3. Chứng minh rằng
	
	Lời giải (của người chấm bài).
	Trong phần trình bày dưới đây, (a, b) kí hiệu ước số chung lớn nhất của hai số nguyên dương a, b.
	Giả sử n có tất cả k ước nguyên tố, kí hiệu là  
	Khi đó,   và                                                                                    (1)
	Do n là số nguyên dương lẻ lớn hơn 1, không chia hết cho 3, nên k  1 và với mọi i = 1, 2, …, k, ta có:
	(2)
	(3)
	Do (3) nên mọi ước nguyên tố của   i = 1, 2, …, k, cũng là ước nguyên tố của               (4)
	Xét số i tùy ý thuộc {1; 2; …; k}.
	Do (2) nên tồn tại   và r \in {1; 2}, sao cho   Do đó
	
	Mà   là số nguyên dương lẻ lớn hơn 9 (do (2)), nên nó phải có ước nguyên tố lớn hơn 3. Vì thế, gọi   là ước nguyên tố lớn nhất của   ta có                                                                      (5)     
	Do   nên   suy ra
	(6)
	Do tính “tùy ý” của i, nên ta có (5) và (6) với mọi i = 1, 2, …, k.
	Từ đó dễ dàng suy ra
	với mọi i, j \in {1; 2; …; k}, i  j.                                              (7)
	Thật vậy, giả sử ngược lại, tồn tại i, j \in {1; 2; …; k}, i  j, sao cho  
	Đặt   theo (5) và (6), ta có q là số nguyên tố lớn hơn 3, và
	(8)
	Dễ thấy,   và   Vì thế, từ (8) suy ra
	
	là điều vô lí (do q > 3). Vì vậy, (7) được chứng minh.
	Do n là số nguyên dương lẻ nên   Kết hợp điều này với (4), (5) và (7), ta được 3,   là các ước nguyên tố đôi một phân biệt của   Vì vậy
	(9)
	Tiếp theo, ta sẽ chứng minh
	với mọi i = 1, 2, …, k.                                                (10)
	Thật vậy, do   với mọi i = 1, 2, …, k, nên theo định lí Fermat nhỏ, ta có:
	(11)
	với mọi i = 1, 2, …, k.
	Lại do   nên từ (6) và (11) suy ra
	(12)
	với mọi i = 1, 2, …, k.
	Do với mọi i = 1, 2, …, k,   là số nguyên tố lẻ và   là số nguyên dương chẵn, nên
	
	với mọi i = 1, 2, …, k.
	Nếu tồn tại i \in {1; 2; …; k} sao cho   thì
	
	mâu thuẫn với (12). Vì vậy
	
	với mọi i = 1, 2, …, k.
	Suy ra,   hay   với mọi i = 1, 2, …, k.
	(10) được chứng minh.
	Từ (9), (10) và (1), suy ra
	
	Ta có điều phải chứng minh theo yêu cầu đề bài.
	Bình luận và Nhận xét
	1. Trong Lời giải trên, có sử dụng (không chứng minh) kết quả quen biết sau:
	“Cho p là một số nguyên tố, và cho a, m, n là các số nguyên dương,   Khi đó, nếu
	
	thì  ”
	Với các bạn đọc chưa biết kết quả nêu trên, các bạn hãy tự chứng minh, xem như một “bài tập về nhà”.
	2. Cho tới thời điểm bản thảo vào Nhà in, Tạp chí vẫn chưa nhận được lời giải nào từ bạn đọc.
	Lưu Thị Thanh Hà
	
\end{multicols}

